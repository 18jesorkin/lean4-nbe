
\documentclass{memoir}

\usepackage{sourcecodepro}
\usepackage{sourcesanspro}
\usepackage{sourceserifpro}

\usepackage{fancyvrb}
\usepackage{fvextra}

\makechapterstyle{lean}{%
\renewcommand*{\chaptitlefont}{\sffamily\HUGE}
\renewcommand*{\chapnumfont}{\chaptitlefont}
% allow for 99 chapters!
\settowidth{\chapindent}{\chapnumfont 999}
\renewcommand*{\printchaptername}{}
\renewcommand*{\chapternamenum}{}
\renewcommand*{\chapnumfont}{\chaptitlefont}
\renewcommand*{\printchapternum}{%
\noindent\llap{\makebox[\chapindent][l]{%
\chapnumfont \thechapter}}}
\renewcommand*{\afterchapternum}{}
}

\chapterstyle{lean}

\setsecheadstyle{\sffamily\bfseries\Large}
\setsubsecheadstyle{\sffamily\bfseries\large}
\setsubsubsecheadstyle{\sffamily\bfseries}

\renewcommand{\cftchapterfont}{\normalfont\sffamily}
\renewcommand{\cftsectionfont}{\normalfont\sffamily}
\renewcommand{\cftchapterpagefont}{\normalfont\sffamily}
\renewcommand{\cftsectionpagefont}{\normalfont\sffamily}

\title{\sffamily Normalization by Evaluation in Lean4}
\author{\sffamily Jeremy Sorkin}
\date{\sffamily }

\begin{document}

\frontmatter

\begin{titlingpage}
\maketitle
\end{titlingpage}

\tableofcontents

\mainmatter

\chapter{Introduction}

For this project, I chose to formalize Peter Dybjer's \hyperlink{"https://www.cs.le.ac.uk/events/mgs2009/courses/courses.html#anchorPeter"}{Normalization-by-Evaluation} slides into Lean4.
My motivation for doing this was for 2 primary reasons:

\begin{itemize}
\item It's an interesting normalization-strategy for rewriting systems that leverages on the interplay between the "Object" and "Meta"-levels in a nontrivial manner.\item It also provides a nice oppurtunity for me to further sharpen and apply my Lean4 skills
\end{itemize}


The goal of this document is to walk the reader through both Dybjer's slides and their Lean4 formalization in a step-by-step manner.
# Motivation

Normalization by Evaluation is a technique whereby we can normalize terms of a given rewrite-system by "evaluating" them into the Meta-level (where normalization occurs)
followed by "reifying" our normalized term back into the Object-level. In this way, we will be translating our "reduction-based" rewriting-relation into a "reduction-free" equality check
that will let us show, among other results, decidability of rewriting.
As a normalization technique, has a history of application to different type systems:

\begin{itemize}
\item \hyperlink{"https://www.mathematik.uni-muenchen.de/~schwicht/papers/lics91/paper.pdf"}{Berger and Schwichtenberg (1991)} utilize Nbe to give a normalization-procedure for Simple Type Theory with βη-rewriting.\item \hyperlink{"https://www.mathematik.uni-muenchen.de/~logik/minlog/"}{MINLOG proof-assistant} utilizes Nbe as a normalization procedure for minimal first order logic.\item \hyperlink{"https://www.cse.chalmers.se/~peterd/papers/GlueingTypes93.pdf"}{Coquand and Dybjer 1993} utilize Nbe to give a decision algorithm for Combinatory logic as well as implement its
    formal correctness-proof in the ALF proof-assistant.\item \hyperlink{"https://tidsskrift.dk/brics/article/view/21870"}{Filinski and Rohde 2005} extended NbE to the Untyped Lambda-Calculus using an infinitary-variant of normal-forms, Bohm Trees.\item \hyperlink{"https://www.cse.chalmers.se/~peterd/papers/NbeMLTTEqualityJudgements.pdf"}{Abel, Dybjer, Coquand 2007} have extended the technique to Martin-Loff Type Theory.
\end{itemize}


For our purposes, we will focus on the simpler-examples of rewriting in Monoids and a Combinatory-version of Godel's System T.




\chapter{Monoid-Rewriting}



\section{Monoid Expressions and rewriting}

Let’s start with the simple case where we are rewriting in a Monoid.
Our Monoid-Expressions are:

\begin{verbatim}
inductive Exp (α : Type)
| app : Exp α → Exp α → Exp α
| id  : Exp α
| elem : α → Exp α
infix : 100 " ⬝ " => Exp.app

\end{verbatim}



That is, an expression is either:

\begin{itemize}
\item Applying two expressions together: \Verb|e₁ ⬝ e₂|\item The identity element \Verb|id|\item An element of \Verb|α|
\end{itemize}


We know that a Monoid has an identity element and is associative, so we naturally get
the following rewrite-relation \Verb|~|:

\begin{verbatim}
inductive convr : (Exp α) → (Exp α) → Prop
| assoc         : convr ((e ⬝ e') ⬝ e'') (e ⬝ (e' ⬝ e''))
| id_left       : convr ((Exp.id) ⬝ e) (e)
| id_right      : convr (e ⬝ Exp.id) (e)
| refl          : convr (e) (e)
| sym           : convr (e) (e') → convr (e') (e)
| trans         : convr (e) (e') → convr (e') (e'') → convr (e) (e'')
| app           : convr (a) (b) → convr (c) (d) → convr (a ⬝ c) (b ⬝ d)
infix : 100 " ~ " => convr

\end{verbatim}



The first 3 constructors are the normal Monoid-laws, the next 3 ensure \Verb|~| is an equivalence-relation, and the final ensures it is congruent wrt \Verb|app|.
This gives us the normal Monoid-behavior we expect as the following examples show:

\begin{verbatim}
def zero := (Exp.id : Exp Nat)
def one  := Exp.elem 1
def two  := Exp.elem 2
def three := Exp.elem 3

-- (1 ⬝ 2) ⬝ ((0 ⬝ 0) ⬝ 3) ~ 1 ⬝ (2 ⬝ (3 ⬝ 0))
theorem example1
        : ((one ⬝  two) ⬝ ((zero ⬝  zero) ⬝  three))
        ~ (one ⬝  (two ⬝  (three ⬝  zero))) :=
  by
  -- Hint:
  -- (1 ⬝ 2) ⬝ ((0 ⬝ 0) ⬝ 3) ~ (1 ⬝ 2) ⬝ (0 ⬝ 3)
  -- (1 ⬝ 2) ⬝ (0 ⬝ 3) ~ (1 ⬝ 2) ⬝ 3
  -- (1 ⬝ 2) ⬝ 3 ~ (1 ⬝ 2) ⬝ (3 ⬝ 0)
  -- (1 ⬝ 2) ⬝ (3 ⬝ 0) ~ 1 ⬝ (2 ⬝ (3 ⬝ 0))
  sorry

\end{verbatim}



\hyperlink{"https://live.lean-lang.org/#codez=JYWwDg9gTgLgBAWQIYwBYBtgCMB0AVJAYxmEICgzgA7AEwFdjgA3AUzgFEAPMOACiXRhUSOAC44eAJ5gWASjIAfOEjA9xXHgKEjASYQduywcLh6Nh7YrjAacMfs1GklluhYg7W46YOenlWgwkrHCEEFRMUHa8Zr6yJnwxjnF6AApQEGCWSADO2RCEcADebCwA5HBl5eo+jgC+dqHhkby8LDgqPGWy7aoVpaVxrT08raXDfQPyStYA+q4AZvDFttUO2vW2KyFhEXzR3DjW3R0Vg3KWs1DAAOaoS2yr5sIbm+KNu0MnGoc0Z1NwUBY83Qm1sy0evheW3ezTkfHOSmykncoKKJSq9ieSA2bx2sLOA3iMPhhNa/xgUCQVGyaL6EzsiXWDTx8IJyW2TRJBMJemJZJJAyyvUKIiwITgNghdVBuM5/EGWHZfMIg1+RJZ/HGKr4uBOv3882AnDsAEYAAxmuAAIjgAD9rXAALwAPg5EQoZBoQLgAC8WOkxI6EgdrAyDIBUQnkXvmcDCbED9hwLjccBNnu9MAA7hAVkHvsn3AAmdMxtCAh55g4FuAAZgoAFp63wTXBALjUcELcXbLUt7bNXdrcXtLe7hbbfBr4/78lcIBAIhYnCQ4FcLbIqNefDj4yzEGOvT2fvS4yPe53qHL8g3tntvG3J14u/Gj4vLDaJ1Psi/gfXcCwkl/Rs4AACWoGBREApteBHDsBx7KcBxrIdm3HTtx14XtB0glD2zQ7tMKQu0cNg8c61sIDoNQxCiMo3C4MnPsrzgCiYLwicEKImDeDHbsGLgadfzyKAoEkIA"}{Try it!}
\hyperlink{"https://live.lean-lang.org/#codez=JYWwDg9gTgLgBAWQIYwBYBtgCMB0AVJAYxmEICgzgA7AEwFdjgA3AUzgFEAPMOACiXRhUSOAC44eAJ5gWASjIAfOEjA9xXHgKEjASYQduywcLh6Nh7YrjAacMfs1GklluhYg7W46YOenlWgwkrHCEEFRMUHa8Zr6yJnwxjnF6AApQEGCWSADO2RCEcADebCwA5HBl5eo+jgC+dqHhkby8LDgqPGWy7aoVpaVxrT08raXDfQPyStYA+q4AZvDFttUO2vW2KyFhEXzR3DjW3R0Vg3KWs1DAAOaoS2yr5sIbm+KNu0MnGoc0Z1NwUBY83Qm1sy0evheW3ezTkfHOSmykncoKKJSq9ieSA2bx2sLOA3iMPhhNa/xgUCQVGyaL6EzsiXWDTx8IJyW2TRJBMJemJZJJAyyvUKIiwITgNghdVBuM5/EGWHZfMIg1+RJZ/HGKr4uBOv3882AnDsAEYAAxmuAAIjgAD9rXAALwAPg5EQoZBoQLgAC8WOkxI6EgdrAyDIBUQnkXvmcDCbED9hwLjccBNnu9MAA7hAVkHvsn3AAmdMxtCAh55g4FuAAZgoAFp63wTXBALjUcELcXbLUt7bNXdrcXtLe7hbbfBr4/78lcIBAIhYnCQ4FcLbIqNefDj4yzEGOvT2fvS4yPe53qHL8g3tntvG3J14u/Gj4vLDaJ1Psi/gfXcCwkl/YRglQS1xF4U8T39Pc7V9KCExhH45iBGBAKQYCWzA8CoMg9J9x4Ms3yHPgIJOAi4VEIMEJOEC3SgHBAWBVDgLHTD716Xc8MPbCPygziyK/X9rxglo2Pw7NOKw49SNfOQ4go2itTxeigRBVA01sIC2FQSdWKod92PE58SPYmSBKEm89lEuAOPPS94KUk4EIYkFeAQ2YFhgK84E07yABYojvPSd0M6S7ME69b0C/SxLPB8yJwvc5MohzeiclS+DcmgZiuW4YBwJEQDiJitIAVgCqybLimSErM8zbysx9s2feKeNw795Ko3J8l/QhXCQSIaLU39f0bZtx07cceynAcayI3gRw7AdeF7Qdfw6dBJAUikqRpVAxxGpt5vGpaVtm4SFom9s61sdbNoQ7bqW8664FGo720u2tzuOybJz7Lzbq2ylHtQXyDrG96lt+uB+xghbeDHbsoenNbVA2wGdu8krf0XIh4DS4EgA"}{Solution}

\begin{verbatim}
-- (0 ⬝ (1 ⬝ 0)) ⬝ ((0 ⬝  2) ⬝ (3 ⬝ 0)) ~ 1 ⬝ (2 ⬝ (3 ⬝ 0))
theorem example2
  : ((zero ⬝ (one ⬝ zero))  ⬝  ((zero ⬝ two) ⬝ (three ⬝ zero)))
  ~ (one ⬝ (two ⬝ (three ⬝ zero))) :=
  by
  -- Hint:
  -- (0 ⬝ (1 ⬝ 0)) ⬝ ((0 ⬝  2) ⬝ (3 ⬝ 0)) ~ (1 ⬝ 0) ⬝ ((0 ⬝  2) ⬝ (3 ⬝ 0))
  -- (1 ⬝ 0) ⬝ ((0 ⬝  2) ⬝ (3 ⬝ 0)) ~ 1 ⬝ ((0 ⬝  2) ⬝ (3 ⬝ 0))
  -- 1 ⬝ ((0 ⬝  2) ⬝ (3 ⬝ 0)) ~ 1 ⬝ (2 ⬝ (3 ⬝ 0))
  sorry

\end{verbatim}



\hyperlink{"https://live.lean-lang.org/#codez=JYWwDg9gTgLgBAWQIYwBYBtgCMB0AVJAYxmEICgzgA7AEwFdjgA3AUzgFEAPMOACiXRhUSOAC44eAJ5gWASjIAfOEjA9xXHgKEjASYQduywcLh6Nh7YrjAacMfs1GklluhYg7W46YOenlWgwkrHCEEFRMUHa8Zr6yJnwxjnF6AApQEGCWSADO2RCEcADebCwA5HBl5eo+jgC+dqHhkby8LDgqPGWy7aoVpaVxrT08raXDfQPyStYA+q4AZvDFttUO2vW2KyFhEXzR3DjW3R0Vg3KWs1DAAOaoS2yr5sIbm+KNu0MnGoc0Z1NwUBY83Qm1sy0evheW3ezTkfHOSmykncoKKJSq9ieSA2bx2sLOA3iMPhhNa/xgUCQVGyaL6EzsiXWDTx8IJyW2TRJBMJemJZJJAyyvUKIiwITgNghdVBuM5/EGWHZfMIg1+RJZ/HGKr4uBOv3882AnDsAEYAAxmuAAIjgAD9rXAALwAPg5EQoZBoQLgAC8WOkxI6EgdrAyDIBUQnkXvmcDCbED9hwLjccBNnu9MAA7hAVkHvsn3AAmdMxtCAh55g4FuAAZgoAFp63wTXBALjUcELcXbLUt7bNXdrcXtLe7hbbfBr4/78lcIBAIhYnCQ4FcLbIqNefDj4yzEGOvT2fvS4yPe53qHL8g3tntvG3J14u/Gj4vLDaJ1Psi/gfXcCwkl/YRglQS1xF4U8T39Pc7V9KCExhH45iBGBAKQYCWzA8CoMg9J9x4Ms3yHPgIJOAi4VEIMEJOEC3SgHBAWBMhG2bcdO3HHspwHGsiN4EcOwHXhe0HVDgLHTD716Xc8MPbCPyg6SyK/X9rxgloJPw7NpKw49SNfOQ4go2itTxeigRBVA02Y3jWIEoTuNUvi2PbOtbCAthUEncSqHfSTNOfEjJL0pSVJvPZ1LgKTz0veCTJOBCGJBXgENmBYYHkKzHK4hybPYyc+yvOA3MKgAWKI728nc/N06LlOvW9yp8jSzwfMicL3AzKNi3p4rMvhkpoGYrluGAcCREA4iYptrPbJyJ04mC+N4MduzyuBpxE9yAFYyvCyKWr0trgpC29wsfbNn1auTcO/QyqNyfJf0IVwkEiGiLN/Dp0EkIyKSpGlUDHD7VC+n7KWpQqXOUYHvoQ37wdQYqgbAEHYbB/7Nt/RciHgHrGMmvghOmtbv27QTx343LOKIxblrm/KZzcecKiXFcWGLLZtIgZ9ws/LTOcq5rehfctDoKk6Kpa879pFq72o639/1/ZiAAlqBgUQlamwm+OndiyfbCmVqp7L8r1oTDbp4mCoy+bSfN2beFW3XhzN8mHado7mMW/XbHd42XdHSn6d/PIoCgSQgA"}{Try it!}
\hyperlink{"https://live.lean-lang.org/#codez=JYWwDg9gTgLgBAWQIYwBYBtgCMB0AVJAYxmEICgzgA7AEwFdjgA3AUzgFEAPMOACiXRhUSOAC44eAJ5gWASjIAfOEjA9xXHgKEjASYQduywcLh6Nh7YrjAacMfs1GklluhYg7W46YOenlWgwkrHCEEFRMUHa8Zr6yJnwxjnF6AApQEGCWSADO2RCEcADebCwA5HBl5eo+jgC+dqHhkby8LDgqPGWy7aoVpaVxrT08raXDfQPyStYA+q4AZvDFttUO2vW2KyFhEXzR3DjW3R0Vg3KWs1DAAOaoS2yr5sIbm+KNu0MnGoc0Z1NwUBY83Qm1sy0evheW3ezTkfHOSmykncoKKJSq9ieSA2bx2sLOA3iMPhhNa/xgUCQVGyaL6EzsiXWDTx8IJyW2TRJBMJemJZJJAyyvUKIiwITgNghdVBuM5/EGWHZfMIg1+RJZ/HGKr4uBOv3882AnDsAEYAAxmuAAIjgAD9rXAALwAPg5EQoZBoQLgAC8WOkxI6EgdrAyDIBUQnkXvmcDCbED9hwLjccBNnu9MAA7hAVkHvsn3AAmdMxtCAh55g4FuAAZgoAFp63wTXBALjUcELcXbLUt7bNXdrcXtLe7hbbfBr4/78lcIBAIhYnCQ4FcLbIqNefDj4yzEGOvT2fvS4yPe53qHL8g3tntvG3J14u/Gj4vLDaJ1Psi/gfXcCwkl/YRglQS1xF4U8T39Pc7V9KCExhH45iBGBAKQYCWzA8CoMg9J9x4Ms3yHPgIJOAi4VEIMEJOEC3SgHBAWBMhG2bcdO3HHspwHGsiN4EcOwHXhe0HVDgLHTD716Xc8MPbCPyg6SyK/X9rxgloJPw7NpKw49SNfOQ4go2itTxeigRBVA02Y3jWIEoTuNUvi2PbOtbCAthUEncSqHfSTNOfEjJL0pSVJvPZ1LgKTz0veCTJOBCGJBXgENmBYYHkKzHK4hybPYyc+yvOA3MKgAWKI728nc/N06LlOvW9yp8jSzwfMicL3AzKNi3p4rMvhkpoGYrluGAcCREA4iYptrPbJyJ04mC+N4MduzyuBpxE9yAFYyvCyKWr0trgpC29wsfbNn1auTcO/QyqNyfJf0IVwkEiGiLN/Dp0EkIyKSpGlUDHD7VC+n7KWpQqXOUYHvoQ37wdQYqgbAEHYbB/7Nt/RciHgHrGMmvghOmtbv27QTx343LOKIxblrm/KZzcecKiXFcWGLLZtIgZ9ws/LTOcq5rehfctDoKk6Kpa879pFq72o639/w2wrQOI2ShZ5+S+YCpqtMu3pedqjdxca2DcOffnSKqoW9Z4XmYqaZ9+qQxY4lx9AlYsqINbNh8Ld8wWRht032sN1F7VOv2dYug7ZZuzqHYfJ2hruV2TISj2xNjCWhcjiKrcDmP9c10PQXD7PA6l63C9tzX7YiYyHYSvqup4J3UqMhKCqswm+OndiyfbCmVqp7L8v7oSh7p4mCs+mGTLh/6zV/DL5tJifZt4Va++HcfyY3rejtn0G/sKtNbGYxaB9sfeR530dKfppGUfntHCvZ5nsY7sygA"}{Solution}

\begin{verbatim}
-- (1 ⬝ 2) ⬝ ((0 ⬝ 0) ⬝ 3) ~ (0 ⬝ (1 ⬝ 0)) ⬝ ((0 ⬝  2) ⬝ (3 ⬝ 0))
theorem example3
  : ((one ⬝ two) ⬝ ((zero ⬝ zero) ⬝ three))
  ~ ((zero ⬝ (one ⬝ zero)) ⬝ ((zero ⬝ two) ⬝ (three ⬝ zero))) :=
  by
  -- Hint: Use examples 1 and 2!
  sorry

\end{verbatim}



\hyperlink{"https://live.lean-lang.org/#codez=JYWwDg9gTgLgBAWQIYwBYBtgCMB0AVJAYxmEICgzgA7AEwFdjgA3AUzgFEAPMOACiXRhUSOAC44eAJ5gWASjIAfOEjA9xXHgKEjASYQduywcLh6Nh7YrjAacMfs1GklluhYg7W46YOenlWgwkrHCEEFRMUHa8Zr6yJnwxjnF6AApQEGCWSADO2RCEcADebCwA5HBl5eo+jgC+dqHhkby8LDgqPGWy7aoVpaVxrT08raXDfQPyStYA+q4AZvDFttUO2vW2KyFhEXzR3DjW3R0Vg3KWs1DAAOaoS2yr5sIbm+KNu0MnGoc0Z1NwUBY83Qm1sy0evheW3ezTkfHOSmykncoKKJSq9ieSA2bx2sLOA3iMPhhNa/xgUCQVGyaL6EzsiXWDTx8IJyW2TRJBMJemJZJJAyyvUKIiwITgNghdVBuM5/EGWHZfMIg1+RJZ/HGKr4uBOv3882AnDsAEYAAxmuAAIjgAD9rXAALwAPg5EQoZBoQLgAC8WOkxI6EgdrAyDIBUQnkXvmcDCbED9hwLjccBNnu9MAA7hAVkHvsn3AAmdMxtCAh55g4FuAAZgoAFp63wTXBALjUcELcXbLUt7bNXdrcXtLe7hbbfBr4/78lcIBAIhYnCQ4FcLbIqNefDj4yzEGOvT2fvS4yPe53qHL8g3tntvG3J14u/Gj4vLDaJ1Psi/gfXcCwkl/YRglQS1xF4U8T39Pc7V9KCExhH45iBGBAKQYCWzA8CoMg9J9x4Ms3yHPgIJOAi4VEIMEJOEC3SgHBAWBMhG2bcdO3HHspwHGsiN4EcOwHXhe0HVDgLHTD716Xc8MPbCPyg6SyK/X9rxgloJPw7NpKw49SNfOQ4go2itTxeigRBVA02Y3jWIEoTuNUvi2PbOtbCAthUEncSqHfSTNOfEjJL0pSVJvPZ1LgKTz0veCTJOBCGJBXgENmBYYHkKzHK4hybPYyc+yvOA3MKgAWKI728nc/N06LlOvW9yp8jSzwfMicL3AzKNi3p4rMvhkpoGYrluGAcCREA4iYptrPbJyJ04mC+N4MduzyuBpxE9yAFYyvCyKWr0trgpC29wsfbNn1auTcO/QyqNyfJf0IVwkEiGiLN/Dp0EkIyKSpGlUDHD7VC+n7KWpQqXOUYHvoQ37wdQYqgbAEHYbB/7Nt/RciHgHrGMmvghOmtbv27QTx343LOKIxblrm/KZzcecKiXFcWGLLZtIgZ9ws/LTOcq5rehfctDoKk6Kpa879pFq72o639/w2wrQOI2ShZ5+S+YCpqtMu3pedqjdxca2DcOffnSKqoW9Z4XmYqaZ9+qQxY4lx9AlYsqINbNh8Ld8wWRht032sN1F7VOv2dYug7ZZuzqHYfJ2hruV2TISj2xNjCWhcjiKrcDmP9c10PQXD7PA6l63C9tzX7YiYyHYSvqup4J3UqMhKCqswm+OndiyfbCmVqp7L8v7oSh7p4mCs+mGTLh/6zV/DL5tJifZt4Va++HcfyY3rejtn0G/sKtNbGYxaB9sfeR530dKfppGUfntHCvZ5nsY7syGymzLd7H5yPEe4jzXnvASB8GZzgXCzZGLBIZgV2vnWwLRtbBwUkFX8t5UENVFubVBe0q4yyLtdH8thFbnybAACWoDAcQABVbIJQYGuBpC2KkNhCwAEJfx5CgFASQQA"}{Try it!}
\hyperlink{"https://live.lean-lang.org/#codez=JYWwDg9gTgLgBAWQIYwBYBtgCMB0AVJAYxmEICgzgA7AEwFdjgA3AUzgFEAPMOACiXRhUSOAC44eAJ5gWASjIAfOEjA9xXHgKEjASYQduywcLh6Nh7YrjAacMfs1GklluhYg7W46YOenlWgwkrHCEEFRMUHa8Zr6yJnwxjnF6AApQEGCWSADO2RCEcADebCwA5HBl5eo+jgC+dqHhkby8LDgqPGWy7aoVpaVxrT08raXDfQPyStYA+q4AZvDFttUO2vW2KyFhEXzR3DjW3R0Vg3KWs1DAAOaoS2yr5sIbm+KNu0MnGoc0Z1NwUBY83Qm1sy0evheW3ezTkfHOSmykncoKKJSq9ieSA2bx2sLOA3iMPhhNa/xgUCQVGyaL6EzsiXWDTx8IJyW2TRJBMJemJZJJAyyvUKIiwITgNghdVBuM5/EGWHZfMIg1+RJZ/HGKr4uBOv3882AnDsAEYAAxmuAAIjgAD9rXAALwAPg5EQoZBoQLgAC8WOkxI6EgdrAyDIBUQnkXvmcDCbED9hwLjccBNnu9MAA7hAVkHvsn3AAmdMxtCAh55g4FuAAZgoAFp63wTXBALjUcELcXbLUt7bNXdrcXtLe7hbbfBr4/78lcIBAIhYnCQ4FcLbIqNefDj4yzEGOvT2fvS4yPe53qHL8g3tntvG3J14u/Gj4vLDaJ1Psi/gfXcCwkl/YRglQS1xF4U8T39Pc7V9KCExhH45iBGBAKQYCWzA8CoMg9J9x4Ms3yHPgIJOAi4VEIMEJOEC3SgHBAWBMhG2bcdO3HHspwHGsiN4EcOwHXhe0HVDgLHTD716Xc8MPbCPyg6SyK/X9rxgloJPw7NpKw49SNfOQ4go2itTxeigRBVA02Y3jWIEoTuNUvi2PbOtbCAthUEncSqHfSTNOfEjJL0pSVJvPZ1LgKTz0veCTJOBCGJBXgENmBYYHkKzHK4hybPYyc+yvOA3MKgAWKI728nc/N06LlOvW9yp8jSzwfMicL3AzKNi3p4rMvhkpoGYrluGAcCREA4iYptrPbJyJ04mC+N4MduzyuBpxE9yAFYyvCyKWr0trgpC29wsfbNn1auTcO/QyqNyfJf0IVwkEiGiLN/Dp0EkIyKSpGlUDHD7VC+n7KWpQqXOUYHvoQ37wdQYqgbAEHYbB/7Nt/RciHgHrGMmvghOmtbv27QTx343LOKIxblrm/KZzcecKiXFcWGLLZtIgZ9ws/LTOcq5rehfctDoKk6Kpa879pFq72o639/w2wrQOI2ShZ5+S+YCpqtMu3pedqjdxca2DcOffnSKqoW9Z4XmYqaZ9+qQxY4lx9AlYsqINbNh8Ld8wWRht032sN1F7VOv2dYug7ZZuzqHYfJ2hruV2TISj2xNjCWhcjiKrcDmP9c10PQXD7PA6l63C9tzX7YiYyHYSvqup4J3UqMhKCqswm+OndiyfbCmVqp7L8v7oSh7p4mCs+mGTLh/6zV/DL5tJifZt4Va++HcfyY3rejtn0G/sKtNbGYxaB9sfeR530dKfppGUfntHCvZ5nsY7syGymzLd7H5yPEe4jzXnvASB8GZzgXCzZGLBIZgV2vnWwLRtbBwUkFX8t5UENVFubVBe0q4yyLtdH8thFbn1/jlUBADBwLXYrTTeI8n5zwdgvD+rMz5wAvvQh+09VLAPpv/a+4CqbMOPuDLGrNCyjWRJjJcxAv7AiAA"}{Solution}



\section{Normalization of Monoid-Expressions}

From the examples above, we can see that showing \Verb|a ~ b| step-by-step can be rather tedious.
When checking this in practice, we simply preform all these steps simultaneously by:

"removing all the \Verb|id|'s, shuffling parentheses to the right, then checking for equality"

Can we implement this normalization strategy by interpreting our Monoid-Expressions in a clever way?



\section{Evaluation}

This is exactly what evaluation does, it interprets our expressions such that normalization happens automatically at the Meta-level.
We will do this by interpreting Monoid-Expressions as functions, the "intended"-meaning:

\begin{itemize}
\item \Verb|app| will be function-composition\item \Verb|id| will be the Identity-function\item \Verb|x| will be \Verb|λ e ↦ x ⬝ e|: Applying \Verb|x| to the left
\end{itemize}


This gives us our evaluation function:

\begin{verbatim}
def eval : (Exp α) → (Exp α → Exp α)
  --               (eval a) ∘ (eval b)
  | Exp.app a b => (λ e => eval a (eval b e))
  --               Identity function
  | Exp.id      => id
  --               λ e ↦ x ⬝ e`
  | Exp.elem x  => (λ e => (Exp.elem x) ⬝ e)

\end{verbatim}



Now, by interpreting Monoid-expressions as function-compositions at the Meta-level,
Lean will automatically normalize these compositions as the following shows:

\begin{verbatim}
-- fun e => Exp.elem 1 ⬝ (Exp.elem 2 ⬝ (Exp.elem 3 ⬝ e))
#reduce eval $ (one ⬝ two) ⬝ ((zero ⬝ zero) ⬝ three)

-- fun e => Exp.elem 1 ⬝ (Exp.elem 2 ⬝ (Exp.elem 3 ⬝ e))
#reduce eval $ (zero ⬝ (one ⬝ zero)) ⬝ ((zero ⬝ two) ⬝ (three ⬝ zero))

-- fun e => Exp.elem 1 ⬝ (Exp.elem 2 ⬝ (Exp.elem 3 ⬝ e))
#reduce eval $ (one ⬝ (two ⬝ (three ⬝ zero)))

\end{verbatim}



From the above examples, we see that the evaluations of

\Verb|(1 ⬝ 2) ⬝ ((0 ⬝ 0) ⬝ 3) ~ (0 ⬝ (1 ~ 0)) ⬝ ((0 ⬝ 2) ⬝ (3 ⬝ 0)) ~ 1 ⬝ (2 ⬝ (3 ⬝ 0))|

are all equal to: \Verb|λ e ↦ 1 ⬝ (2 ⬝ (3 ⬝ e))|.

That is, rewritable-terms give us equal evaluations:

\begin{verbatim}
-- a ~ b  → eval a = eval b
theorem convr_eval_eq {a b : Exp α} (h : a ~ b) : ∀ c, (eval a) c  = (eval b) c :=
    by
    -- Hint: Induction on h
    sorry

\end{verbatim}



\hyperlink{"https://live.lean-lang.org/#codez=JYWwDg9gTgLgBAWQIYwBYBtgCMB0AVJAYxmEICgyBaSuAMSghAC45UYYwBnJgeh8M45gEQjgCmYnmIDunATwBMABiUBOHgBMAnlgBWYqDxBJgAOwAKSdCE5jKUMQDdgtjTjAaAZhTMaArsTAjmJwAKIAHmBwABSAjcBwLHhaYGIAlGQAPnBIYFEsEVHxgEmEYZFwxaWFmXDAGnAJleXVYuhiIA0VBU1mnsDhDQCMKnAARHCAuNSjcAC8AHyVODlgFFQ0ACK1pgDk8BpihOhIDnBoIQDKYjAQtTWmnDBIpoQhntAnqC5wWpdkvgEkwTghAgpkcUAa0S6sVScBKkLK0NhcHMDGWWSQnE4IjgAG8QmItnACYT8giAL4NYGg8HRaIhSYEmEMrZbGF0iYxAkc4mpdJZWoAfVanngePqpMKFPq4qBILBMXhYCEGiZRLZaWqgqgwAA5mxcSEJeUpdKWFT5ezJgVleq+XAHJ50NL6mKjbETTLzTS0py7ZwtO1nQaiSTGu7PXLverWUivZyY3S7TAoI9OMGucSGlCKWbI774zCSnG6QmebG84n46zqktcUg4FggXA6m6PbnqTEkGysIXZR3ooQ2Sry/365NBzFG5MVb9TL1+iwhkopgA/KZzPtglZ7TxExxWCFQ3uK8pIo9kepZa21+uNjfRQDdwESZvMnAf63T907G2l0peFjcQYbrUF5wFekTiK07T9C+MRPiE97Wi0bRwOEqoahQ1DZHAa6Nkib5OvW0x7geWBkFBxiblAAoETRACOdYNlm5IxKgDT1rhMIsIAAERAgANJyX7ZDChD1MRn6kSJCTTKBWBaKBfyBCCrCgVY6ACjqEBWGmBxiEc2FkaBhAYiEGJYqJt5NnMoH1GYyYQM2NlwFheDas4VhOViUBQPJRkmTUGhCmIIrYdZzp2QwDZOS5bnAB5zpeT5FD1MZtgBQK2p6vARGzE5EUOWRzoxUEcXoJ50BJX5aUOoRL55aY9lRUVNCuSV8XSolvkpf5/rtJZSAZQKjbAKgYXSsV7llQlFVdUC/nJqmBlNgN1GNlgg2iSNAw1KgChjfUE2leV3mzalpm5Etol1AKcA3Ug60jUCgWPftQKtPpABUznZA9bGEM9qD1Y1YhOX4c4QOgdQEdFNAABJ2SwADCFX7DA6BaLc9yPCQKCmb92SmHU/0Co9x1JUAA"}{Try it!}
\hyperlink{"https://live.lean-lang.org/#codez=JYWwDg9gTgLgBAWQIYwBYBtgCMB0AVJAYxmEICgyBaSuAMSghAC45UYYwBnJgeh8M45gEQjgCmYnmIDunATwBMABiUBOHgBMAnlgBWYqDxBJgAOwAKSdCE5jKUMQDdgtjTjAaAZhTMaArsTAjmJwAKIAHmBwABSAjcBwLHhaYGIAlGQAPnBIYFEsEVHxgEmEYZFwxaWFmXDAGnAJleXVYuhiIA0VBU1mnsDhDQCMKnAARHCAuNSjcAC8AHyVODlgFFQ0ACK1pgDk8BpihOhIDnBoIQDKYjAQtTWmnDBIpoQhntAnqC5wWpdkvgEkwTghAgpkcUAa0S6sVScBKkLK0NhcHMDGWWSQnE4IjgAG8QmItnACYT8giAL4NYGg8HRaIhSYEmEMrZbGF0iYxAkc4mpdJZWoAfVanngePqpMKFPq4qBILBMXhYCEGiZRLZaWqgqgwAA5mxcSEJeUpdKWFT5ezJgVleq+XAHJ50NL6mKjbETTLzTS0py7ZwtO1nQaiSTGu7PXLverWUivZyY3S7TAoI9OMGucSGlCKWbI774zCSnG6QmebG84n46zqktcUg4FggXA6m6PbnqTEkGysIXZR3ooQ2Sry/365NBzFG5MVb9TL1+iwhkopgA/KZzPtglZ7TxExxWCFQ3uK8pIo9kepZa21+uNjfRQDdwESZvMnAf63T907G2l0peFjcQYbrUF5wFekTiK07T9C+MRPiE97Wi0bRwOEqoahQ1DZHAa6Nkib5OvW0x7geWBkFBxiblAAoETRACOdYNlm5IxKgDT1rhMIsIAAERAgANJyX7ZDChD1MRn6kSJCTTGQWBaKBfyBCCrAUPUjxaAKOoQFYnCgfUZjJhAnAANzZGIWLLKBhAYiEN5MaJdQCnATlIFgArAGxhAaO5bFzHpQKtEccAAFRwDQrk+UC3kef5BkMES/mcCkhDAFYwAAF4hF5kViIlyWpZgmXZG5HkKgRzYwhqzqcKAUQgugWhwAA2gRAC6/lQNIzXZR57XOp1zURb1QA"}{Solution}

In addition, we will need the following lemma relating \Verb|eval| to \Verb|app|:

\begin{verbatim}
-- ∀ b, a ⬝ b ~ (eval a b)
theorem app_eval (a : Exp α) : ∀ b, (a ⬝ b) ~ (eval a b) :=
    by
    -- Hint: Induction on a
    sorry

\end{verbatim}



\hyperlink{"https://live.lean-lang.org/#codez=JYWwDg9gTgLgBAWQIYwBYBtgCMB0AVJAYxmEICgyBaSuAMSghAC45UYYwBnJgeh8M45gEQjgCmYnmIDunATwBMABiUBOHgBMAnlgBWYqDxBJgAOwAKSdCE5jKUMQDdgtjTjAaAZhTMaArsTAjmJwAKIAHmBwABSAjcBwLHhaYGIAlGQAPnBIYFEsEVHxgEmEYZFwxaWFmXDAGnAJleXVYuhiIA0VBU1mnsDhDQCMKnAARHCAuNSjcAC8AHyVODlgFFQ0ACK1pgDk8BpihOhIDnBoIQDKYjAQtTWmnDBIpoQhntAnqC5wWpdkvgEkwTghAgpkcUAa0S6sVScBKkLK0NhcHMDGWWSQnE4IjgAG8QmItnACYT8giAL4NYGg8HRaIhSYEmEMrZbGF0iYxAkc4mpdJZWoAfVanngePqpMKFPq4qBILBMXhYCEGiZRLZaWqgqgwAA5mxcSEJeUpdKWFT5ezJgVleq+XAHJ50NL6mKjbETTLzTS0py7ZwtO1nQaiSTGu7PXLverWUivZyY3S7TAoI9OMGucSGlCKWbI774zCSnG6QmebG84n46zqktcUg4FggXA6m6PbnqTEkGysIXZR3ooQ2Sry/365NBzFG5MVb9TL1+iwhkopgA/KZzPtglZ7TxExxWCFQ3uK8pIo9kepZa21+uNjfRQDdwESZvMnAf63T907G2l0peFjcQYbrUF5wFekTiK07T9C+MRPiE97Wi0bRwOEqoahQ1DZHAa6Nkib5OvW0x7geWBkFBxiblAAoETRACOdYNlm5IxKgDT1rhMIsIAAERAgANJyX7ZDChD1MRn6kSJCTTGQWBaKBfyBCCrCgY8WgCjqEBWJwoH1GYyYQJwADc2RiFiyygYQGIhDeTGiXUApwI5SBYAKwBsYQGhuWxcy6UCrRHHAABUcA0C53lAl57l+fpDBEn5nApIQwBWMAABeISeRFYgJUlKWYBl2Sue5CoEc2MIas6nCgFEILoFocAANoEQAun5UDSE1WXuW1zodU14U9ZhNC8VgAljkxa4SYRDbpBR9ZLDRQnRPWbpcXAo0CStHI9jhgnvrN0myfJemmP4SmmNklnWdkuRNnUhARVFPmzDFpgGQ2fl+HOEDoHUBF+VhAAS+kNAAwtADjEPVtz3I8JAoJlEWPHUz0JZD8nXbYNR1L5zpYXg2rOFY6NQFAmP1FZ2PIe0RGvfjNCE0E+Wk+TQA"}{Try it!}
\hyperlink{"https://live.lean-lang.org/#codez=JYWwDg9gTgLgBAWQIYwBYBtgCMB0AVJAYxmEICgyBaSuAMSghAC45UYYwBnJgeh8M45gEQjgCmYnmIDunATwBMABiUBOHgBMAnlgBWYqDxBJgAOwAKSdCE5jKUMQDdgtjTjAaAZhTMaArsTAjmJwAKIAHmBwABSAjcBwLHhaYGIAlGQAPnBIYFEsEVHxgEmEYZFwxaWFmXDAGnAJleXVYuhiIA0VBU1mnsDhDQCMKnAARHCAuNSjcAC8AHyVODlgFFQ0ACK1pgDk8BpihOhIDnBoIQDKYjAQtTWmnDBIpoQhntAnqC5wWpdkvgEkwTghAgpkcUAa0S6sVScBKkLK0NhcHMDGWWSQnE4IjgAG8QmItnACYT8giAL4NYGg8HRaIhSYEmEMrZbGF0iYxAkc4mpdJZWoAfVanngePqpMKFPq4qBILBMXhYCEGiZRLZaWqgqgwAA5mxcSEJeUpdKWFT5ezJgVleq+XAHJ50NL6mKjbETTLzTS0py7ZwtO1nQaiSTGu7PXLverWUivZyY3S7TAoI9OMGucSGlCKWbI774zCSnG6QmebG84n46zqktcUg4FggXA6m6PbnqTEkGysIXZR3ooQ2Sry/365NBzFG5MVb9TL1+iwhkopgA/KZzPtglZ7TxExxWCFQ3uK8pIo9kepZa21+uNjfRQDdwESZvMnAf63T907G2l0peFjcQYbrUF5wFekTiK07T9C+MRPiE97Wi0bRwOEqoahQ1DZHAa6Nkib5OvW0x7geWBkFBxiblAAoETRACOdYNlm5IxKgDT1rhMIsIAAERAgANJyX7ZDChD1MRn6kSJCTTGQWBaKBfyBCCrCgY8WgCjqEBWJwoH1GYyYQJwADc2RiFiyygYQGIhDeTGiXUApwI5SBYAKwBsYQGhuWxcy6UCrRHHAABUcA0C53lAl57l+fpDBEn5nApIQwBWMAABeISeRFYgJUlKWYBl2Sue5CoEc2MIas6nCgFEILoFocAANoEQAun5UDSE1WXuW1zodU14U9ZhNC8VgAljkxa4SYRDbpBR9ZLDRQnRPWbpcXAo0CStHI9jhgnvrN0myfJemmP4SmmNklnWdkuRNnUhARVFPmzDFpgGQ2fl+HOEDoHUBG5fs+XpSEz2fVVeWpYVj0ldNzazX5qBIICqDLiwtKidOqq7VNmMxHUkw9lxxFeosmIiO1Yi9KYIQDpGODJqmrDLgA/AKf7SgcYiBajiPIyEqMQnj0QE7NMK4xycN1ET0lUYsd109SOAOugQ7eRz9QOmYtOk4zdzM3AbMa/53PgrzzpiOERDwDDqAUPUVm2DUdS+c6sUQOD0qW9bcuCsKMD20CN3Ie0rvSu7nv1N7xByyrQA"}{Solution}



\section{Reification}

Now that we can evaluate out Monoid-expressions such that they're normalized at the Meta-level, how do we bring them back down to the object-level such that we don't change the "behavior" (wrt \Verb|~|)?
Well, for a given \Verb|e : Exp α|, we intuitively know that \Verb|eval e : Exp α → Exp α| will have the form:

\Verb|λ e' ↦ elem x₁ ⬝ (elem x₂ ⬝ ... ⬝ (elem xₙ ⬝ e'))|

So to reify it back down while retaining its \Verb|~|-behavior, simply apply \Verb|Exp.id| to the end:

\begin{verbatim}
def reify (f : Exp α → Exp α) : (Exp α) := f Exp.id

\end{verbatim}





\section{Nbe}

With our two main functions in place, we are finally ready to define our \Verb|nbe|-algorithm:

\begin{verbatim}
def nbe (e : Exp α) : Exp α := reify (eval e)

\end{verbatim}



What \Verb|nbe| does is first evaluate \Verb|e| so it gets normalized through function-composition, and then reify's it back into a canonical element of \Verb|[e]~|.
Thus, we can translate \Verb|a ~ b| into the decidable-problem \Verb|nbe a = nbe b| which we state as our main correctness-theorem:

\begin{verbatim}
theorem correctness (e e' : Exp α) : (e ~ e') ↔ (nbe e = nbe e') :=
    by sorry

\end{verbatim}



\hyperlink{"https://live.lean-lang.org/#codez=JYWwDg9gTgLgBAWQIYwBYBtgCMB0AVJAYxmEICgyBaSuAMSghAC45UYYwBnJgeh8M45gEQjgCmYnmIDunATwBMABiUBOHgBMAnlgBWYqDxBJgAOwAKSdCE5jKUMQDdgtjTjAaAZhTMaArsTAjmJwAKIAHmBwABSAjcBwLHhaYGIAlGQAPnBIYFEsEVHxgEmEYZFwxaWFmXDAGnAJleXVYuhiIA0VBU1mnsDhDQCMKnAARHCAuNSjcAC8AHyVODlgFFQ0ACK1pgDk8BpihOhIDnBoIQDKYjAQtTWmnDBIpoQhntAnqC5wWpdkvgEkwTghAgpkcUAa0S6sVScBKkLK0NhcHMDGWWSQnE4IjgAG8QmItnACYT8giAL4NYGg8HRaIhSYEmEMrZbGF0iYxAkc4mpdJZWoAfVanngePqpMKFPq4qBILBMXhYCEGiZRLZaWqgqgwAA5mxcSEJeUpdKWFT5ezJgVleq+XAHJ50NL6mKjbETTLzTS0py7ZwtO1nQaiSTGu7PXLverWUivZyY3S7TAoI9OMGucSGlCKWbI774zCSnG6QmebG84n46zqktcUg4FggXA6m6PbnqTEkGysIXZR3ooQ2Sry/365NBzFG5MVb9TL1+iwhkopgA/KZzPtglZ7TxExxWCFQ3uK8pIo9kepZa21+uNjfRQDdwESZvMnAf63T907G2l0peFjcQYbrUF5wFekTiK07T9C+MRPiE97Wi0bRwOEqoahQ1DZHAa6Nkib5OvW0x7geWBkFBxiblAAoETRACOdYNlm5IxKgDT1rhMIsIAAERAgANJyX7ZDChD1MRn6kSJCTTGQWBaKBfyBCCrCgY8WgCjqEBWJwoH1GYyYQJwADc2RiFiyygYQGIhDeTGiXUApwI5SBYAKwBsYQGhuWxcy6UCrRHHAABUcA0C53lAl57l+fpDBEn5nApIQwBWMAABeISeRFYgJUlKWYBl2Sue5CoEc2MIas6nCgFEILoFocAANoEQAun5UDSE1WXuW1zodU14U9ZhNC8VgAljkxa4SYRDbpBR9ZLDRQnRPWbpcXAo0CStHI9jhgnvrN0myfJemmP4SmmNklnWdkuRNnUhARVFPmzDFpgGQ2fl+HOEDoHUBG5fs+XpSEz2fVVeWpYVj0ldNzazX5qBIICqDLiwtKidOqq7VNmMxHUkw9lxxFeosmIiO1Yi9KYIQDpGODJqmrDLgA/AKf7SgcYiBajiPIyEqMQnj0QE7NMK4xycN1ET0lUYsd109SOAOugQ7eRz9QOmYtOk4zdzM3AbMa/53PgrzzpiOERDwDDqAUPUVm2DUdS+c6sUQOD0qW9bcuCsKMD20CN3Ie0rvSu7nv1N7xByyrFA7vaYjAJ4DXRLubpngi60noiTDEbu1ogWQCemFgtOGmG61QrLDjJ6nZUYVhZyoBAsjNkDGgucAmAwA1EC7iEa4EmQpzQChwJQA4xA05inIhsxhTZ4PIYwoAKYQxKX+IzHAm8r0dcmqbk9VwAAkp4nhCO9DCgQA7bcH1227Z3/MIl2P9KakaVp6A6UGpnmX5R2IRwDyyiLee62QBTUUbI9aiDknIvj8g7AK4IQphSgQKGBGCNBINCjQAAEvpFgABVJ2XolpWHorgrEk95K3zgN9V4f0d5l0TnXN6D8lB+SwoQ96JCnaC0eHURaAMqrQCgFoIAA"}{Try it!}
\hyperlink{"https://live.lean-lang.org/#codez=JYWwDg9gTgLgBAWQIYwBYBtgCMB0AVJAYxmEICgyBaSuAMSghAC45UYYwBnJgeh8M45gEQjgCmYnmIDunATwBMABiUBOHgBMAnlgBWYqDxBJgAOwAKSdCE5jKUMQDdgtjTjAaAZhTMaArsTAjmJwAKIAHmBwABSAjcBwLHhaYGIAlGQAPnBIYFEsEVHxgEmEYZFwxaWFmXDAGnAJleXVYuhiIA0VBU1mnsDhDQCMKnAARHCAuNSjcAC8AHyVODlgFFQ0ACK1pgDk8BpihOhIDnBoIQDKYjAQtTWmnDBIpoQhntAnqC5wWpdkvgEkwTghAgpkcUAa0S6sVScBKkLK0NhcHMDGWWSQnE4IjgAG8QmItnACYT8giAL4NYGg8HRaIhSYEmEMrZbGF0iYxAkc4mpdJZWoAfVanngePqpMKFPq4qBILBMXhYCEGiZRLZaWqgqgwAA5mxcSEJeUpdKWFT5ezJgVleq+XAHJ50NL6mKjbETTLzTS0py7ZwtO1nQaiSTGu7PXLverWUivZyY3S7TAoI9OMGucSGlCKWbI774zCSnG6QmebG84n46zqktcUg4FggXA6m6PbnqTEkGysIXZR3ooQ2Sry/365NBzFG5MVb9TL1+iwhkopgA/KZzPtglZ7TxExxWCFQ3uK8pIo9kepZa21+uNjfRQDdwESZvMnAf63T907G2l0peFjcQYbrUF5wFekTiK07T9C+MRPiE97Wi0bRwOEqoahQ1DZHAa6Nkib5OvW0x7geWBkFBxiblAAoETRACOdYNlm5IxKgDT1rhMIsIAAERAgANJyX7ZDChD1MRn6kSJCTTGQWBaKBfyBCCrCgY8WgCjqEBWJwoH1GYyYQJwADc2RiFiyygYQGIhDeTGiXUApwI5SBYAKwBsYQGhuWxcy6UCrRHHAABUcA0C53lAl57l+fpDBEn5nApIQwBWMAABeISeRFYgJUlKWYBl2Sue5CoEc2MIas6nCgFEILoFocAANoEQAun5UDSE1WXuW1zodU14U9ZhNC8VgAljkxa4SYRDbpBR9ZLDRQnRPWbpcXAo0CStHI9jhgnvrN0myfJemmP4SmmNklnWdkuRNnUhARVFPmzDFpgGQ2fl+HOEDoHUBG5fs+XpSEz2fVVeWpYVj0ldNzazX5qBIICqDLiwtKidOqq7VNmMxHUkw9lxxFeosmIiO1Yi9KYIQDpGODJqmrDLgA/AKf7SgcYiBajiPIyEqMQnj0QE7NMK4xycN1ET0lUYsd109SOAOugQ7eRz9QOmYtOk4zdzM3AbMa/53PgrzzpiOERDwDDqAUPUVm2DUdS+c6sUQOD0qW9bcuCsKMD20CN3Ie0rvSu7nv1N7xByyrFA7vaYjAJ4DXRLubpngi60noiTDEbu1ogWQCemFgtOGmG61QrLDjJ6nZUYVhZyoBAsjNkDGgucAmAwA1EC7iEa4EmQpzQChwJQA4xA05inIhsxhTZ4PIYwoAKYQxKX+IzHAm8r0dcmqbk9VwAAkp4nhCO9DCgQA7bcH1227Z3/MIl2P9KakaVp6A6UGpnmX5R2IRwDyyiLee62QBTUUbI9aiDknIvj8g7AK4IQphSgQKGBGCNBINYPzbIsDMEtk7ByCcU0pzlVlqTG8GCsFwNwdHG2kYlpWHogQ2hdRC44NvnAb6rw/o7zLonOub0H5KD5ijAYDRl6WgAsOPOMRSb+naKTLUuo2DGy1jTRR9M9ZplQFIo2gCUGsAGBIgWUj0b0jkeLUqQkfRcOJrdMALCnQhC4ZTamOtdEpn1gYw27NjGm1MZTaQ2oYAhEaqjXqnMTHm2lEjFGaM7EHkZDYvaJYOSOKofTZRnZciuPno4zx2sdFKz0QbIxzouY83ERbK2MdVFeW1HqGAQA"}{Solution}

With \Verb|correctness| in place, Lean can now instantly decide any \Verb|a ~ b| problem through reflexivity, i.e.:

\begin{verbatim}
-- (1 ⬝ 2) ⬝ ((0 ⬝ 0) ⬝ 3) ~ 1 ⬝ (2 ⬝ (3 ⬝ 0))
theorem example1'
        : (one.app two).app  ((zero.app zero).app three)
        ~ (one.app (two.app (three.app zero))) :=
  by
  /- Try this:
    exact (correctness ((one ⬝ two) ⬝ ((zero ⬝ zero) ⬝ three)) (one ⬝ (two ⬝ (three ⬝ zero)))).mpr rfl
  -/
  exact?

-- (0 ⬝ (1 ⬝ 0)) ⬝ ((0 ⬝  2) ⬝ (3 ⬝ 0)) ~ 1 ⬝ (2 ⬝ (3 ⬝ 0))
theorem example2'
  : (zero.app (one.app zero)).app ((zero.app two).app (three.app zero))
  ~ (one.app (two.app (three.app zero))) :=
  by
  /- Try this:
    exact (correctness ((zero ⬝ (one ⬝ zero)) ⬝ ((zero ⬝ two) ⬝ (three ⬝ zero)))
      (one ⬝ (two ⬝ (three ⬝ zero)))).mpr rfl
  -/
  exact?

-- (1 ⬝ 2) ⬝ ((0 ⬝ 0) ⬝ 3) ~ (0 ⬝ (1 ⬝ 0)) ⬝ ((0 ⬝  2) ⬝ (3 ⬝ 0))
theorem example3'
  : (one.app two).app  ((zero.app zero).app three)
  ~ (zero.app (one.app zero)).app ((zero.app two).app (three.app zero)) :=
  by
  /- Try this:
    exact (correctness ((one ⬝ two) ⬝ ((zero ⬝ zero) ⬝ three))
      ((zero ⬝ (one ⬝ zero)) ⬝ ((zero ⬝ two) ⬝ (three ⬝ zero)))).mpr rfl
  -/
  exact?

\end{verbatim}






\chapter{GodelT-Rewriting}



\section{GodelT Expressions and rewriting}

We now move on to a Combinatory-version of Godel's System T and implement NbE for it as well.
Here we will be using an Intrinsic encoding (aka "typing ala Church") meaning all Expressions will be well-typed and we won't need an additional "Derivation" type.
First, we define our Simple-Types:

\begin{verbatim}
inductive Ty : Type
| Nat : Ty
| arrow : Ty → Ty → Ty
open Ty
infixr : 100 " ⇒' " => arrow

\end{verbatim}



Here our base-type is the Natural Numbers while \Verb|arrow| lets us form functions between Simple Types.
Our Expressions are:

\begin{verbatim}
inductive ExpT : Ty → Type
| K {a b : Ty}     :  ExpT (a ⇒' b ⇒' a)
| S {a b c : Ty}   :  ExpT ((a ⇒' b ⇒' c) ⇒' (a ⇒' b) ⇒' (a ⇒' c))
| App {a b : Ty}   :  ExpT (a ⇒' b) → ExpT a → ExpT b
| zero             :  ExpT .Nat
| succ             :  ExpT (.Nat ⇒' .Nat)
| recN {a : Ty}    :  ExpT (a ⇒' (.Nat ⇒' a ⇒' a) ⇒' .Nat ⇒' a)
open ExpT
infixl : 100 " ⬝ " => App

\end{verbatim}



That is, our Expressions are either:

\begin{itemize}
\item Combinators \Verb|K| and \Verb|S|\item Applying two expressions together: \Verb|e₁ ⬝ e₂|\item The Natural number \Verb|zero|\item The successor function \Verb|succ|\item The recursor for Natural numbers: \Verb|recN|
\end{itemize}


From this, we get the following rewriting relation for \Verb|~|:

\begin{verbatim}
inductive convrT : (ExpT α) → (ExpT α) → Prop
| refl  : convrT (e) (e)
| sym   : convrT (e) (e') → convrT (e') (e)
| trans : convrT (e) (e') → convrT (e') (e'') → convrT (e) (e'')
| K     : convrT (K ⬝ x ⬝ y) (x)
| S     : convrT (S ⬝ x ⬝ y ⬝ z) (x ⬝ z ⬝ (y ⬝ z))
| app   : convrT (a) (b) → convrT (c) (d) → convrT (a ⬝ c) (b ⬝ d)
| recN_zero : convrT (recN ⬝ e ⬝ f ⬝ .zero) (e)
-- TO-DO: Fix Verso so it lets me define this constructor!!
--| recN_succ : convrT (recN ⬝ e ⬝ f ⬝ (.succ ⬝ n)) (f ⬝ n ⬝ (recN ⬝ e ⬝ f ⬝ n))
infix : 100 " ~ " => convrT

\end{verbatim}



Some rewriting examples are:

\begin{verbatim}
-- Identity Combinator
def I {β : Ty} : ExpT (α ⇒' α) := (@ExpT.S α (β ⇒' α) α) ⬝ .K ⬝ .K

example {x : ExpT α}
        : ((@I α β) ⬝ x)  ~  x :=
  by
  /-
  apply convrT.trans convrT.S ?_
  apply convrT.trans convrT.K ?_
  exact convrT.refl
  -/
  -- TO-DO: Try-it! Here
  sorry

\end{verbatim}



\begin{verbatim}
-- B Combinator
def B {a b c : Ty} : ExpT ((b ⇒' c) ⇒' (a ⇒' b) ⇒' a ⇒' c) := .S ⬝ (.K ⬝ .S) ⬝ .K


example {x : ExpT (b ⇒' c)}
        {y : ExpT (a ⇒' b)}
        {z : ExpT a}
        : (B ⬝ x ⬝ y ⬝ z)  ~  (x ⬝ (y ⬝ z)) :=
  by
  /-
  unfold B
  apply convrT.trans ?_ ?_
  exact (.K ⬝ .S) ⬝ x ⬝ (.K ⬝ x) ⬝ y ⬝ z
  · apply convrT.app ?_ convrT.refl
    apply convrT.app ?_ convrT.refl
    exact convrT.S
  · apply convrT.trans ?_ ?_
    exact .S ⬝ (.K ⬝ x) ⬝ y ⬝ z
    · apply convrT.app ?_ convrT.refl
      apply convrT.app ?_ convrT.refl
      apply convrT.app ?_ convrT.refl
      exact convrT.K
    · apply convrT.trans ?_ ?_
      exact (.K ⬝ x ⬝ z) ⬝ (y ⬝ z)
      · exact convrT.S
      · apply convrT.app ?_ convrT.refl
        exact convrT.K
  -/
  -- TO-DO: Try-it! here
  sorry

\end{verbatim}



\begin{verbatim}
def add (m n : ExpT .Nat) := .recN ⬝ m ⬝ (.K ⬝ .succ) ⬝ n

example : (add m .zero)  ~  m :=
  by
  /-
  exact convrT.recN_zero
  -/
  -- TO-DO: "Try-it!" here
  sorry

example : (add m (.succ ⬝ n))  ~  (.succ ⬝ (add m n)) :=
  by
  /-
  unfold add
  apply convrT.trans ?_ ?_
  exact (.K ⬝ .succ) ⬝ n ⬝ (.recN ⬝ m ⬝ (.K ⬝ .succ) ⬝ n)
  · sorry --exact convrT.recN_succ
  · apply convrT.app ?_ convrT.refl
    exact convrT.K
  -/
  -- TO-DO: "Try-it!" here
  sorry

\end{verbatim}





\section{Normalizing Godel-T Expressions}

In this instance, implementing Evaluation and Reification will be trickier than the previous case because
we will need to evaluate Types and Terms and Term-Evaluation will depend on reification.
So, we will implement it in the following order:

\begin{itemize}
\item Evaluate Simple Types to Meta-Level Types\item Reify Meta-level Expressions to Godel-T Expressions\item Evaluate Godel-T Expressions to Meta-Level Expressions
\end{itemize}


So now for our first question in implementing NbE: how will we evaluate Simple Types to the Meta-level?



\section{Evaluating Simple Types}

To do this, we will make use of Tait's reducibility method and a constructive proof of Weak Normalization to
extract our NbE algorithm.

With the Reducibility method, we define "Reducibility" in such a way that strengthens our Inductive Hypothesis
and allows us to prove Weak Normalization. The main idea is: functions will be considered reducible if they map
reducible inputs to reducible outputs while base terms will be considered reducible if they are Weakly-Normalizing.

From this, we get the following definition for Reducibility:

\Verb|Red_{Nat}(e)           = WN_{Nat}(e)|

\Verb|Red_{alpha -> beta}(f) = WN_{alpha -> beta}(f) & Forall x : alpha, Red_{alpha}(x) -> Red_{beta}(f x)|

Now, to get our constructive NbE algorithm, we are going to remove all intermediate proof terms (witnessing Weak-Normalization)
from our \Verb|Red| relation and only keep the returned Expressions.

Doing this gives us our Evaluation Function for Simple Types:

\begin{verbatim}
def Ty_inter : Ty → Type
| Ty.Nat => Nat

| Ty.arrow a b => ExpT (a ⇒' b) × (Ty_inter a → Ty_inter b)

\end{verbatim}



Here:

\begin{itemize}
\item For \Verb|Ty.Nat|: we give it the standard interpretation of Lean's Natural Numbers\item For \Verb|a ⇒' b|: given a \Verb|f : ExpT (a ⇒' b)|, we "glue" onto it a function \Verb|F : (Ty_inter a → Ty_inter b)|
    describing how \Verb|f| behaves when applied to normalized-arguments. This will be more clear when we evaluate Godel-T Expressions.
\end{itemize}




\section{Reification and Semantic-Application}

In order to Evaluate Godel-T terms, we must first describe how to reify Meta-level terms back to Godel-T terms.
This is because, when defining the glued-on semantic function \Verb|F : (Ty_inter a → Ty_inter b)| we will need reification to be able
to apply \Verb|f : ExpT (a ⇒' b)| to reified semantic arguments.

Defining reification in this instance is simple because of how we evaluate types:

\begin{itemize}
\item For \Verb|n : Ty_inter Ty.Nat = ℕ|: Return the standard numeral \Verb|succ ⬝ succ ⬝ ... ⬝ zero  :  ExpT Ty.Nat|\item For \Verb|(f, F) : Ty_inter (a ⇒' b) = ExpT (a ⇒' b) × (Ty_inter a → Ty_inter b)|: Return \Verb|f : ExpT (a ⇒' b)|
\end{itemize}


This give us the following reification function:

\begin{verbatim}
def reifyT (a : Ty) (e : Ty_inter a) : ExpT a :=
/-
| Ty.Nat, (0 : ℕ) => zero
| Ty.Nat, n+1     => succ ⬝ (reify Ty.Nat n)

| Ty.arrow a b, (f, F) => f
-/
by
  cases a
  case Nat =>
    induction e
    case zero           => exact .zero
    case succ n reify_n => exact (.succ ⬝ reify_n)
  case arrow a b        => exact e.fst

\end{verbatim}



In addition, we will need to define application on the Meta-level where our glued-on function \Verb|F : (Ty_inter a → Ty_inter b)|
comes into play:

\begin{itemize}
\item For \Verb|(f, F) : Ty_inter (a ⇒' b) = ExpT (a ⇒' b) × (Ty_inter a → Ty_inter b)| and \Verb|e' : Ty_inter a|:  Return \Verb|F e' : Ty_inter b|
\end{itemize}


which gives us:

\begin{verbatim}
def appsem {a b : Ty} (t : Ty_inter (a ⇒' b)) (e' : Ty_inter a) : Ty_inter b := (t.snd e')

\end{verbatim}





\section{Evaluating Godel-T Expressions}

With all of the constituents in place, we are finally able to Evaluate Godel-T Expressions to Meta-level terms.
Before giving the formal definition, we first describe how the Evaluation works on an example to give some intuition.

For \Verb|zero : ExpT Ty.Nat| and \Verb|succ : ExpT (Ty.Nat ⇒' Ty.Nat)|, these will get their standard translations of \Verb|0 : ℕ| and \Verb|λ n : ℕ ↦ n+1|.

For a generic function \Verb|f : ExpT (α =>' β)|, we are going to keep applying \Verb|f| to reified semantic-arguments until
\Verb|f ⬝ (reify x1) ⬝ ... ⬝ (reify xn)| is fully-applied, then we will use \Verb|f|'s standard-translation on \Verb|x1, ..., xn|. This will in-effect
capture all of \Verb|f|'s \Verb|~|-behavior into one big tuple which is how \Verb|F : Ty_inter a → Ty_inter b| works.

We illustrate this with the exampe \Verb| S :  ExpT ((a ⇒' b ⇒' c) ⇒' (a ⇒' b) ⇒' (a ⇒' c))|.

The first step is to evalute \Verb|S|'s type to the Meta-level, for readibility let:

\Verb|τ := (a ⇒' b ⇒' c) ⇒' (a ⇒' b) ⇒' (a ⇒' c)|

\Verb|τ' := (a ⇒' b) ⇒' (a ⇒' c)|

\Verb|τ'' := a ⇒' c|

and

\Verb|⟦ ⬝ ⟧ := Ty_inter|

Then we get:

\Verb|⟦τ⟧ = ExpT τ × (⟦a ⇒' b ⇒' c⟧ → ⟦τ'⟧)|

\Verb|   = ExpT τ × (⟦a ⇒' b ⇒' c⟧ → (ExpT τ' × (⟦a ⇒' b⟧ → ⟦τ''⟧)))|

\Verb|   = ExpT τ × (⟦a ⇒' b ⇒' c⟧ → (ExpT τ' × (⟦a ⇒' b⟧ → (ExpT τ'' × (⟦a⟧ → ⟦c⟧)))))|

Now for describing each of the components: let \Verb|x : ⟦a ⇒' b ⇒' c⟧, y : ⟦a ⇒' b⟧, z : ⟦a⟧|

\begin{itemize}
\item \Verb|ExpT τ|: This is \Verb|S| applied to no arguments, so \Verb|S : ExpT τ|\item \Verb|ExpT τ'|: This is \Verb|S| applied to 1 reified-argument, so \Verb|S ⬝ (reify x) : ExpT τ'|\item \Verb|ExpT τ''|: This is \Verb|S| applied to 2 reified-arguments, so \Verb|S ⬝ (reify x) ⬝ (reify y) : ExpT τ''|\item \Verb|⟦a⟧ → ⟦c⟧|: This is \Verb|S| fully-applied to 3 reified-arguments, so \Verb|S ⬝ (reify x) ⬝ (reify y) ⬝ (reify z) : ⟦c⟧|.
  However, since \Verb|S| is fully-applied, we may "semantically-unfold" this one-step to get: \Verb|appsem (appsem x z) (appsem y z) : ⟦c⟧|
\end{itemize}


Putting this all together, we get the semantic-evaluation for \Verb|S|:

\Verb|ExpT_inter S|
\Verb|=|

\Verb|(S,|

\Verb|(λ x ↦ (S ⬝ (reify (a⇒'b⇒'c) x),|

\Verb|(λ y ↦ (S ⬝ (reify (a⇒'b⇒'c) x) ⬝ (reify (a⇒'b) y),|

\Verb|(λ z ↦ appsem (appsem x z) (appsem y z)))))))|

Through this, we can see that our evaluation of GodelT expressions is to keep applying them to reified-arguments
until we can reduce it at the Meta-level.

Repeating this for our other constructors finally gives us Term-evaluation:

\begin{verbatim}
def ExpT_inter (a : Ty) : (e : ExpT a) → Ty_inter a
| @ExpT.K a b => (.K,
            (λ p ↦ (.K ⬝ (reifyT a p),
            (λ q ↦ p))))

| @ExpT.S a b c => (.S,
              (λ x ↦ (.S ⬝ (reifyT (a⇒'b⇒'c) x),
              (λ y ↦ (.S ⬝ (reifyT (a⇒'b⇒'c) x) ⬝ (reifyT (a⇒'b) y),
              (λ z ↦ appsem (appsem x z) (appsem y z)))))))
| @ExpT.App a b f e  => appsem (ExpT_inter (a ⇒' b) f) (ExpT_inter a e)

| _ => sorry

-- TO-DO: Fix Verso so I can implement the rest of this Definition!!
/-
| ExpT.zero          => (0 : Nat)

| ExpT.succ          => (.succ,
                   (λ n : Nat ↦ n+1) )
| @ExpT.recN a       => (.recN,
                   (λ p ↦ (.recN ⬝ (reifyT a p),
                   (λ q ↦ (.recN ⬝ (reifyT a p) ⬝ (reifyT (.Nat⇒'a⇒'a) q),
                   (λ n0 ↦ Nat.rec p (λ n r ↦ appsem (appsem q n) r) n0))))))
-/

\end{verbatim}








\chapter{Index}






\end{document}
